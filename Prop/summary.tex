% Note: must be in third person, must reflect both ``intellectual
% merit and broader impact'' of the proposal.  ``objectives and
% methods''

%% From the NSF: The proposal must contain a summary of the proposed
%% activity suitable for publication, not more than one page in
%% length.  It should not be an abstract of the proposal, but rather a
%% self-contained description of the activity that would result if the
%% proposal were funded.  The summary should be written in the third
%% person and include a statement of objectives and methods to be
%% employed.  It must clearly address in separate statements (within
%% the one-page summary): (1) the intellectual merit of the proposed
%% activity; and (2) the broader impacts resulting from the proposed
%% activity.  (See Chapter III for further descriptive information on
%% the NSF merit review criteria.)  It should be informative to other
%% persons working in the same or related fields and, insofar as
%% possible, understandable to a scientifically or technically
%% literate lay reader.  Proposals that do not separately address both
%% merit review criteria within the one page Project Summary will be
%% returned without review.

%% State the "objectives", "goals" or "challenges" of this work

\documentclass[12pt]{article}

\topmargin=-0.5in
\oddsidemargin=0in
\evensidemargin=0in
\textwidth=6.5in
\textheight=9.0in

\usepackage{color}
\usepackage{xspace}

\newcommand{\Propane}{{\sc Propane}\@\xspace}
\newcommand{\Name}{{\sc Butane}\@\xspace}

\begin{document}
\setcounter{page}{1}

 \begin{large}
\begin{center}
NeTS: Medium: Collaborative Research: \\
Network Configuration Synthesis: A Path to Practical Deployment
\end{center}
\end{large}

\noindent
\textbf{Overview:} 
Surveys of network operators have shown that human error during
network configuration and update is one of the top causes of network
downtime.  
%% One fundamental reason for these misconfigurations is the
%% mismatch between the intended high-level
%% policies of operators and the low-level, device-by-device configurations
%% that implement them.  On the one hand, most routing policies involve
%% network-wide properties:  never announce a particular destination externally, isolate two subnetworks from one another, prefer traffic through a transit customer over a transit provider, etc.
%% On the other hand, these policies must be implemented via hundreds of low-level configuration directives at each individual router in the network. 
%% As a result, errors range from simple consistency issues (community values do not match) to more subtle scenarios (misconfigured ACLs causing intermittent disconnectivity).  Moreover, networks networks must continue to function properly when node and link failures occur---a near impossible task when reasoning
%% about network behavior without automated help. 
Recently, researchers have proposed a number of
high-level abstractions for programmatic network control, as well
as compiler algorithms that synthesize the nitty-gritty details
of router configurations.  The PIs recent work on \Propane 
(SIGCOMM 2016 Best Paper Award) is a promising effort in this direction
as it allows users to describe end-to-end paths for intra- and
inter-domain routing, as well as the back-up paths needed in the case of 
failure.  Unfortunately, many barriers to practical deployment of this 
and related technologies persist as current languages are insufficiently
expressive, lack tools that make their outputs intelligeable, 
provide no help migrating legacy configurations to new models, and
do not admit incremental deployment.
%(1) they are insufficiently \emph{expressive},
%lacking abstractions that match operator cognitive models; (2) their
%output configurations are insufficiently \emph{intelligeable}, preventing
%operators from auditing or debugging their outputs; (3) they provide
%no tools to help \emph{migrate} from legacy configurations to new
%abstractions; and (4) they do not admit \emph{incremental deployment}.

\noindent
\textbf{Intellectual Merit:} 
The goal of this project is to 
{\em surmount the technical challenges that impede practical 
deployment of high-level network abstractions}.  We will demonstrate
our advances by building on \Propane and delivering a new system \Name, which
will exhibit:  
\begin{enumerate}
\item {\bf New topological abstractions}:  Users will be able to declare device 
\emph{roles} (\emph{e.g.}, top-of-rack switch)
and the \emph{connectivity invariants} related to them.  A new compiler
will verify safety properties of policies in the presence of such 
declarations and generate parameterized \emph{templates}
that make compiler outputs more \emph{intelligeable} for operators.  
\item {\bf Costs and contracts for inter-domain transit:}
Users will specify the 
\emph{financial transit contracts} that govern transit
costs using a new declarative, sub-language and our compiler will
optimize inter- and intra-domain routes for them automatically.
\item {\bf Backend diversity:}  We will develop algorithms that
can synthesize and exploit the relative benefits of heterogenous back-ends 
involving BGP, OSPF, OpenFlow and P4.
%In order to give users access to
%the benefits of performance properties of different protocols,
%we will develop algorithms that synthesize configurations employing a 
%combination of \emph{inter-domain routing} protocols such as BGP 
%and \emph{intra-domain routing protocols} such as OSPF and RIP as well
%as OpenFlow and P4. 
\item {\bf Migration technology:} We will develop
tools that will help network operators \emph{infer} 
new high-level configurations from existing low-level configurations
and to \emph{verify} that new configurations are equivalent to old ones.
\item {\bf Mixed operation and verification:} We will support mixed
legacy- and \Name-managed network operations so engineers can
migrate their networks slowly over time and test partial deployment on
small fractions of their live traffic.
\end{enumerate}

\noindent
\textbf{Broader Impacts:} 
%Our economy, businesses, governmental and 
%military infrastructure all depend upon having networks that function
%reliably. This proposal will tackle the technical challenges that
%prevent deployment of high-level programming abstractions capable
%of improving network reliability.  In addition, 
We will 
\emph{work with
operators of two major clouds}, Microsoft Azure and Google
to test our language and system using the policies of real industrial networks, to identify the pragmatic barriers to adoption, and to deploy our system where possible.   We will also engage in more general \emph{industry outreach}
by organizing a workshop under
the umbrella of the newly-formed
Cornell-Princeton Center for Network Programming (CNP).
% in which 
%academics and industrial researchers exchange problems, solutions,
%ideas and contact information.  
Our \emph{educational plan} involves
developing new curriculum at UCLA with the aim of producing
future network engineers with interdisciplinary skills in verification
and programming languages.  
We will also work to provide both
graduate and undergraduate
\emph{under-represented minorities} research opportunities as we
have successfully done in the past.

\medskip
\noindent\textbf{Keywords:} network programming languages, 
domain-specific programming languages, program synthesis, 
network verification, network management, inter-domain routing

\end{document}


%%% Local Variables:
%%% mode: latex
%%% TeX-master: "proposal.tex"
%%% TeX-PDF-mode: t
%%% End:
