% Note: must be in third person, must reflect both ``intellectual
% merit and broader impact'' of the proposal.  ``objectives and
% methods''

%% From the NSF: The proposal must contain a summary of the proposed
%% activity suitable for publication, not more than one page in
%% length.  It should not be an abstract of the proposal, but rather a
%% self-contained description of the activity that would result if the
%% proposal were funded.  The summary should be written in the third
%% person and include a statement of objectives and methods to be
%% employed.  It must clearly address in separate statements (within
%% the one-page summary): (1) the intellectual merit of the proposed
%% activity; and (2) the broader impacts resulting from the proposed
%% activity.  (See Chapter III for further descriptive information on
%% the NSF merit review criteria.)  It should be informative to other
%% persons working in the same or related fields and, insofar as
%% possible, understandable to a scientifically or technically
%% literate lay reader.  Proposals that do not separately address both
%% merit review criteria within the one page Project Summary will be
%% returned without review.

%% State the "objectives", "goals" or "challenges" of this work


\begin{center}\large\bf
SHF: Small: Collaborative Research:  \\
Higher-order Type-theoretic Program Synthesis with Examples
\end{center}

One of the great bottlenecks in many aspects of scientific discovery, in 
economic development, in the functioning of our government, and even in 
completing repetitive, personal, day-to-day tasks is the rate at which
we can write programs that automate our work and analyze data.  
One key way to alleviate this
bottleneck is to develop fundamental new techniques to \textit{synthesize} the 
programs we need to accomplish these tasks.

\textbf{Intellectual Merit:}  Over the last several years, there has
been tremendous progress in the synthesis of pure, first-order
functions from a remarkably small number of examples.  Such progress
has led to high-profile industrial successes such as Gulwani's FlashFill~\cite{flashfill},
which transforms Excel spreadsheet data, and PowerShell~\cite{powershell},
which parses structured text files given examples.  
Reuseable synthesis platforms such 
Sketch~\cite{sketch}, a synthesizer for first-order, C-like programs, have
demonstrated their utility by supporting applications such as X, Y, and Z! 
(Z is so great!!)

\dpw{re-write this to focus on higher-order, polymorphic and refinement types, 
I think.}
However, in general, these synthesis techniques have been limited in
scope to the synthesis of \emph{first-order} And \emph{effect-free} 
programs.  In contrast, we propose a new research program that will
tackle the problem of how to synthesis \emph{higher-order} and
\emph{effectful} programs in their full generality.  The key to doing
so, we believe, is to exploit not only examples, but also \emph{type theory}
in the synthesis process.  To illustrate the potential
of our ideas, we have already built an initial prototype
system that uses novel algorithms driven by structure of 
\emph{simple} (recursive) types and is capable of synthesizing
higher-order functions given only a small number of examples.
Our proposed research will involve developing new data structures
and algorithms to improve the scaling properties of our prototype
and to investigate how to incorporate a broad range of more advanced
type theories in to our synthesis algorithms.  These new type theories
will include 
(1) \emph{polymorphic types} to learn more general functions, 
(2) \emph{dependent types} to further constrain functions where specification
via examples is inefficient, 
(3) \emph{monadic types} to encapsulate simple effects such as exceptions, 
(4) \emph{linear types} (to encapsulate resource utilization effects such as 
memory allocation), and 
(5) \emph{advanced type-and-effect systems} to encapsulate more general
kinds of effect specifications.
In order to evaluate our ideas in practice, we propose to develop a new
general-purpose, higher-order, effectful and typed programming language,
called \OSynth{}, based around OCaml, which incorporates synthesis
features in to the language.


\dpw{This paragraph perhaps shouldn't be in the summary? Move to intro?}
The two PIs are uniquely qualified to carry out the proposed research.
Both PIs are experts in type systems and logic.  PI Zdancewic has been
developing a pure type-and-example-based system that will form the
initial kernel of the proposed work.  PI Walker brings experience in
data-driven synthesis from his past NSF-sponsored work on the PADS
project~\cite{pads,pads-synth}.

The PIs have worked together successfully in the past and the
collaboration will get off to a quick start with PI Walker will be on
sabbatical at Pi Zdancewic's institution (University of Pennsylvania)
in the 2015-2016 academic year.

\medskip
\noindent\textbf{Keywords:} linear logic, linear algebra, probability,
statistics, denotational semantics, nonstandard models, programming
languages


%%% Local Variables:
%%% mode: latex
%%% TeX-master: "proposal.tex"
%%% TeX-PDF-mode: t
%%% End:
