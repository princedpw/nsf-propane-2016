\documentclass[11pt]{article}

\topmargin=-0.5in
\oddsidemargin=0in
\evensidemargin=0in
\textwidth=6.5in
\textheight=9.0in


\begin{document}
  
\section*{Todd D. Millstein}
\vspace{-.15in}
Computer Science Department\\
4532K Boelter Hall\\
University of California, Los Angeles, CA  90095-1596\\
(310) 825-5942 [Voice], (310) 794-5057 [Fax], todd@cs.ucla.edu [Email]

\subsection*{Professional Preparation}

Brown University, Providence, RI, Computer Science, {\bf
  A.B., 1996.}

\noindent
University of Washington, Seattle, WA, Computer Science,
{\bf M.S., 1998.} 

\noindent
University of Washington, Seattle, WA, Computer Science,
{\bf Ph.D., 2003.}

\subsection*{Appointments}

\begin{tabular}{ll}
7/15--present & Professor, University of California, Los Angeles. \\
7/09--6/15 & Associate Professor, University of California, Los Angeles. \\
7/12--9/12 & Visiting Researcher, Microsoft Research, Redmond.\\
7/10--1/11 & Academic Visitor, University of Oxford.\\
11/03--6/09 & Assistant Professor, %% Computer Science Department,
University of California, Los Angeles. 
% 9/96--8/03 &  Research and Teaching Assistant, %% Department of Computer
% %% Science \& Engineering,
% University of Washington.
\end{tabular}

\paragraph{Closely Related Products}
\begin{itemize}

\item Seyed K. Fayaz,
              Tushar Sharma,
              Ari Fogel,
              Ratul Mahajan,
              Todd Millstein,
              Vyas Sekar, and
              George Varghese.  From Header Space to Control Space: Comprehensive Network Reachability Verification.  {\em USENIX Symposium on Operating Systems Design and Implementation} (OSDI 2016), to appear, 2016.

\item Ryan Beckett, Ratul Mahajan, Todd Millstein, Jitendra Padhye, and David Walker.  Don't Mind the Gap: Bridging Network-wide Objectives and Device-level Configurations.  {\em ACM SIGCOMM Conference} (SIGCOMM 2016), pages 328--341, 2016.

\item Ari Fogel, Stanley Fung, Luis Pedrosa, Meg Walraed-Sullivan, Ramesh Govindan, Ratul Mahajan, and Todd Millstein.  A General Approach to Network Configuration Analysis.  {\em USENIX Symposium on Networked Systems Design and Implementation} (NSDI 2015), pages 469--483, 2015.

\item Luis Pedrosa, Ari Fogel, Nupur Kothari, Ramesh Govindan, Ratul Mahajan, and 
Todd Millstein.  Analyzing Protocol Implementations for Interoperability.  {\em USENIX Symposium on Networked Systems Design and Implementation} (NSDI 2015), pages 485--498, 2015.

\item Nupur Kothari, Ratul Mahajan, Todd Millstein, Ramesh Govindan, and Madanlal Musuvathi.  Finding Protocol Manipulation Attacks.  {\em ACM SIGCOMM Conference} (SIGCOMM 2011), pages 26--37, 2011.

\end{itemize}

\paragraph{Other Significant Products}
\begin{itemize}

\item Daniel Marino, Abhayendra Singh, Todd D. Millstein, Madanlal
  Musuvathi, Satish Narayanasamy.
 A Case for an SC-Preserving Compiler. {\em ACM SIGPLAN Conference on
   Programming Language Design and Implementation} (PLDI 2011), pages
 199--210, 2011.

\item Shane Markstrum, Daniel Marino, Matthew Esquivel, Todd
  Millstein, Chris Andreae, and James Noble.
JavaCOP: Declarative Pluggable Types for Java.
{\em ACM Transactions on Programming Languages and
  Systems} (TOPLAS), 32(2):1--37, January 2010. 

\item Sorin Lerner, Todd Millstein, Erika Rice, and Craig Chambers.
  Automated Soundness Proofs for Dataflow Analyses and Transformations
  via Local Rules.  {\em Symposium on Principles of Programming
    Languages} (POPL 2005), pages 364--377, 2005.

\item Sorin Lerner, Todd Millstein, and Craig Chambers.
  Automatically Proving the Correctness of Compiler Optimizations.
  %% In Proceedings of the
  {\em 
ACM SIGPLAN
Conference on Programming Language Design and Implementation} (PLDI
'03), pages 220--231, 2003.

\item Thomas Ball, Rupak Majumdar, Todd Millstein, and Sriram
  K. Rajamani.
  Automatic Predicate Abstraction of C Programs.
%%   In Proceedings of the
  {\em
ACM SIGPLAN
Conference on Programming Language Design and Implementation} (PLDI
'01), pages 203--213, 2001.

\end{itemize}

\subsection*{Synergistic Activities}
\begin{itemize}  
\item Co-developed the predicate abstractor component
of the SLAM software model checker,
which is part of the Static Driver Verifier, a toolkit for validating Windows device
drivers.

% \item Co-developed MultiJava, an extension to Java
%   supporting open classes and multiple dispatch.  Our
% MultiJava compiler {\em mjc} is freely available for download at
% {\tt www.multijava.org} and has been used by others at Intel
% Research, Seattle and the University of Washington to build
%   ubiquitous-computing applications.

\item Program Chair for OOPSLA 2014.

\item Co-developed CamlBack, an online tutoring system for OCaml and
  Haskell that has been used by several universities.

\item Co-developed RERAN, a record-and-replay tool for Android applications.
 RERAN is freely available for download and has been used by others
 both in industry and academia.

% \item Co-developed JavaCOP, a framework for
%   programmer-defined type systems in Java programs.
%   JavaCOP is freely available for download and has been used by other researchers.

\item Co-developed JL5, an extension to the Polyglot compiler
  framework to support Java 5, which is now part of the main Polyglot
  release. 
JL5 has been used by many other researchers.

% \item Co-developed Extensible ML (EML), an object-oriented extension to ML.
%   EML has been used to teach object-oriented programming to
%   undergraduates at Carnegie Mellon University.


% \item Program Committee Chair for the PLPV 2009 Workshop, PLDI 2008 Student Research
%   Competition, and OOPSLA 2008 Doctoral Symposium.  
% Program Committee member for ECOOP 2010, OOPSLA 2009, ECOOP 2008, AOSD 2008, 
% ICFP 2007, % FOAL 2007, 
% OOPSLA 2006, PLDI 
% 2006. % FOAL
% % 2006,
% % PASTE 2005, TLDI 2005, FOAL 2004.

% \item Member of two NSF grant panels.
\end{itemize}

% Reviewer for TOPLAS, PLDI, POPL, OOPSLA, ICFP, JFP, ECOOP, CAV, SIGMOD.


\end{document}
