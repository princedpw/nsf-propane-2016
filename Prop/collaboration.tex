\documentclass[12pt]{article}

\topmargin=-0.5in
\oddsidemargin=0in
\evensidemargin=0in
\textwidth=6.5in
\textheight=9.0in


\begin{document}
 \begin{large}
\begin{center}
Collaboration Plan
\end{center}
\end{large}
\setcounter{page}{1}

Profs.\ Walker, Millstein, and Varghese have already been successfully collaborating with one another for the past few years in the areas of network programming and verification.  Walker and Millstein collaborated on the Propane network programming language; Walker and Varghese collaborated on the P4 language for packet processing; and Millstein and Varghese collaborated on the ERA tool for verifying network control planes.  Through these existing collaborations, the team already has a well honed working relationship. 

A key strength of our proposal is the diversity of expertise among the PIs, ranging from programming languages to automated verification to network algorithmics.  As we have done for our prior and ongoing collaborations, we will continue to hold weekly team-wide videoconference meetings.  This mechanism allows everyone to stay abreast of the activities at both universities and enables all students to take advantage of the broad knowledge and perspectives represented across the team.  In this way, the PIs truly co-advise all graduate students, regardless of their location, which in our experience produces stronger, higher-impact research.  

In addition to the weekly meetings, we have frequent email discussions between meetings.  Further, Millstein and Varghese have nearby offices and interact almost daily.
Finally, we will also hold an annual retreat {\bf do we want to do this?} for more in-depth interactions to assess our research progress and plan next steps, and we will meet in person periodically at the conferences where we present our work.

Due to our existing collaborations we already have a shared Github repository for code, data, and publications.  We will continue to use this repository to share and maintain our research artifacts.

Given the backgrounds and interests of the PIs, we envision a natural
split in the research work.  {\bf Here we should briefly assign the various research tasks to different members of the team (and their students) as being primarily responsible for them, along with one of those annoying timeline diagrams.  Here's a recent example.}

\begin{figure}[!hbt]
\begin{small}
\centerline{\vbox{\offinterlineskip\tabskip=0pt
\def\d{\omit\leaders\hrule height 5pt depth -3pt \hfill}\halign {\strut#
\hfil&&\vrule width 1.0pt#&\hfil\ # &\vrule#&\hfil\ # & \vrule#&\hfil\ # &
	\vrule#&\hfil\ # \cr &&\multispan7\hfil Year 1\hfil&&\multispan7\hfil Year 2\hfil&& \multispan7\hfil Year 3\hfil && \multispan7\hfil Year 4 \hfil &&  
	\cr &&1&&2&&3&&4&&1&&2&&3&&4&&1&&2&&3&&4&&1&&2&&3&&4&&\cr
\noalign{\hrule}
\textit{Research Task}
&&&&&&&&&&&&&&&&&&&&&&&&&&&&&&&&&&\cr
\noalign{\hrule}
Task 1: Big Data Debugging Primitives &&\d&&\d&&\d&&\d&&\d&&\d&&&&&&&&&&&&&   &&&&&&&&&\cr
Task 2: Data Provenance &&\d&&\d&&\d&&\d&&\d&&\d&&\d&&\d&&\d&&\d&&&&&&&&&&&&&&\cr
Task 3: Incremental Computation &&&&&&&&\d&&\d&&\d&&\d&&\d&&\d&&\d&&\d&&\d&&&&&&&&&&\cr
Task 4: Tool-Assisted Fault Localization and Repair &&&&&&&&&&&&&&\d&&\d&&\d&&\d&&\d&&\d&&&&&&&&&&\cr
Task 5: Interaction Log Analysis &&\d&&\d&&\d&&\d&&\d&&\d&&&&&&&&&&&&&&&&&&&&&&\cr
Task 6: User Studies &&&&\d&&\d&&\d&&&&&&&&&&&&&&\d&&\d&&\d&&\d&&\d&&\d&&\cr
Task 7: Evaluation &&&&&&&&&&&&&&&&&&&&&&\d&&\d&&\d&&\d&&\d&&\d&&\cr
\noalign{\hrule}
\noalign{\hrule}}}}
\end{small}
\caption{Research Time line}\label{fig:timeline}
\end{figure} 
\end{document}